\documentclass{article}
\usepackage[utf8]{inputenc}
\usepackage{amsmath, amsfonts, amsthm}

\theoremstyle{plain}
\newtheorem{theorem}{Theorem}
\newtheorem{lemma}[theorem]{Lemma}
\newtheorem{corollary}[theorem]{Corollary}
\newtheorem{proposition}[theorem]{Proposition}
\theoremstyle{definition}
\newtheorem{definition}[theorem]{Definition}
\newtheorem{example}[theorem]{Example}
\newtheorem{counter}[theorem]{Counterexample}
\theoremstyle{remark}
\newtheorem{remark}[theorem]{Remark}
\numberwithin{theorem}{section}

\title{EF-Note}
\author{Dan Marsden}

\begin{document}

\maketitle

\section{Most recent strategy}
This strategy only retains the most recent moves in a play.
\begin{definition}
We define inductively a family of functions on lists over a set~$X$, indexed by a finite subsets of~$X$.
\begin{align*}
    n^X [] &= []\\
    n^X (xs:x) &= 
    \begin{cases}
    n^X xs &\mbox{ if } x \in X\\
    (n^{X \cup \{ x \} }xs) : x &\mbox{ otherwise}
    \end{cases}
\end{align*}
Here for convenience we assume that our lists are snoc lists, so they build up from the right. The notation~$(xs:x)$ is a list~$xs$ with the element~$x$ added at the right.
\end{definition}
\begin{lemma}
$n^\emptyset$ preserves non-empty lists.
\end{lemma}
\begin{proof}
Obvious, as the tail of the list cannot appear in the empty set.
\end{proof}

\end{document}
