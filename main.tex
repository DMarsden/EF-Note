\documentclass{article}
\usepackage[utf8]{inputenc}
\usepackage{amsmath, amsfonts, amsthm}
\usepackage{eqproof}
\usepackage{hyperref}

\usepackage{tikz-cd}

\theoremstyle{plain}
\newtheorem{theorem}{Theorem}
\newtheorem{lemma}[theorem]{Lemma}
\newtheorem{corollary}[theorem]{Corollary}
\newtheorem{proposition}[theorem]{Proposition}
\theoremstyle{definition}
\newtheorem{definition}[theorem]{Definition}
\newtheorem{example}[theorem]{Example}
\newtheorem{counter}[theorem]{Counterexample}
\theoremstyle{remark}
\newtheorem{remark}[theorem]{Remark}
\numberwithin{theorem}{section}

\title{$\mathbb{E}_k$ without repeats v3.0}
\author{Dan Marsden}

\begin{document}

\maketitle

\section{Introduction}
This note looks at a few candidate schemes for eliminating the need for I-morphisms by adjusting the comonad~$\mathbb{E}_k$. Actually, we will simply focus on the question of making non-repeating non-empty sequences into a comonad on relational structures, and not directly look at the relationship with I-morphisms.

We begin by fixing some useful notation:
\begin{itemize}
    \item $\epsilon$ will always be the function taking the tail element of a list, and~$\delta$ will be the non-empty prefixes function.
    \item $L_k$ will denote the~$k$-bounded list functor on sets and functions, and~$L^+_k$ its non-empty counterpart.
    \item We will treat lists as cons lists when convenient, and write~$x:xs$ for list~$xs$ with element~$x$ added at the head.
    \item We will also treat lists as snoc lists when convenient, and write~$xs:x$ for list~$xs$ with element~$x$ added at the tail.
\end{itemize}
To simplify notation, we shall occasionally implicitly overload functions so that they can be applied element wise to tuples of values.

\subsection*{The Proposals}
The note investigates three plans of attack. The aim is to either show they work out, or to explicitly record the deficiencies that cause problems.
\begin{itemize}
    \item Section~\ref{sec:naive} looks at a very naive scheme that fails on basic checking.
    \item Section~\ref{sec:-most-recent-appearance} looks at a better motivated scheme that addresses the main issue with the naive scheme. Unfortunately this also fails early checking.
    \item Section~\ref{sec:whack-a-mole} looks at a somewhat ad-hoc scheme, again addressing the main failing in the naive approach. This tentatively seems to work out, although this seems slightly surprising as superficially it seems less well motivated.
\end{itemize}

\section{The Naive Scheme}
\label{sec:naive}
In the naive scheme, we restrict our functor to non-repeating sequences. We will need to normalize sequences after the functor action. The simple suggestion is to use the following function:
\begin{definition}[Shrinking]
\label{def:shrinking}
We define inductively a family of \emph{shrinking} functions on lists over a set~$X$, indexed by finite subsets of~$X$
\begin{align*}
    s^X [] &= []\\
    s^X (x:xs) &=
    \begin{cases}
    s^X xs &\mbox{if} x \in X\\
    x : s^{X \cup \{ x \}} xs &\mbox{otherwise}
    \end{cases}
\end{align*}
\end{definition}
We define the structure for our proposed endofunctor.
\begin{definition}
For $\sigma$-structure~$A$, define~$\mathbb{E}^*_K$ as having:
\begin{description}
\item[Universe] The universe is the set of non-repeating, non-empty lists of elements from~$A$.
\item[Relations] For n-tuple $\overline{a}$ of elements, and n-ary relation symbol~$R$, we define:
\begin{equation*}
    R^{\mathbb{E}^*_k}(\overline{a})
\end{equation*}
if the elements of~$\overline{a}$ are pairwise comparable in the prefix order, and:
\begin{equation*}
    R^A(\epsilon(\overline{a}))
\end{equation*}
\end{description}
For homomorphism~$h : A \rightarrow B$ we define function:
\begin{equation*}
    \mathbb{E}^*_k(h) = s^\emptyset \circ L^+_k h
\end{equation*}
\end{definition}
This scheme is unfortunately flawed as we observe:
\begin{lemma}
Even at the level of underlying sets, the following diagram does not in general commute:
\begin{equation*}
\begin{tikzcd}
\mathbb{E}^*_k A \ar[r, "\epsilon"] \ar[d, "\mathbb{E}^*_k h"] & A \ar[d, "h"]\\
\mathbb{E}^*_k B \ar[r, "\epsilon"] &  B
\end{tikzcd}
\end{equation*}
\begin{proof}
We consider list~$[a,b,c]$ and mapping:
\begin{align*}
    h(a) &= a\\
    h(b) &= b\\
    h(c) &= a
\end{align*}
Then chasing around the top leads to~$a$. On the other hand, chasing along the bottom leads to~$b$. The essential problem is that~$s^\emptyset$ does not preserve the last element of lists, so normalization breaks things.
\end{proof}

\end{lemma}

\section{Most Recent Appearance Scheme}
\label{sec:-most-recent-appearance}
This scheme only retains the most recent moves in a play, which is well motivated and straightforward to describe.
\begin{definition}
We define inductively a family of functions on lists over a set~$X$, indexed by a finite subsets of~$X$.
\begin{align*}
    n^X [] &= []\\
    n^X (xs:x) &= 
    \begin{cases}
    n^X xs &\mbox{ if } x \in X\\
    (n^{X \cup \{ x \} }xs) : x &\mbox{ otherwise}
    \end{cases}
\end{align*}
\end{definition}
\begin{lemma}
The function $n^X$ has the following properties:
\begin{itemize}
    \item $n^\emptyset$ preserves non-empty lists.
    \item $n^X$ does not contain any elements of~$X$.
\end{itemize}
\end{lemma}
We define the structure for our proposed endofunctor.
\begin{definition}
For $\sigma$-structure~$A$, define~$\mathbb{E}^*_K$ as having:
\begin{description}
\item[Universe] The universe is the set of non-repeating, non-empty lists of elements from~$A$.
\item[Relations] For n-tuple $\overline{a}$ of elements, and n-ary relation symbol~$R$, we define:
\begin{equation*}
    R^{\mathbb{E}^*_k}(\overline{a})
\end{equation*}
if the elements of~$\overline{a}$ are pairwise comparable in the prefix order, and:
\begin{equation*}
    R^A(\epsilon(\overline{a}))
\end{equation*}
\end{description}
For homomorphism~$h : A \rightarrow B$ we define function:
\begin{equation*}
    \mathbb{E}^*_k(h) = n^\emptyset \circ L_k h
\end{equation*}
\end{definition}
So far, hopefully we have not done much more than try and formalize the proposed construction. Unfortunately we quickly hit a hurdle:
\begin{lemma}
For homomorphism~$h : A \rightarrow B$ function~$\mathbb{E}^*_k h$ is not necessarily a homomorphism of type:
\begin{equation*}
    \mathbb{E}^*_k A \rightarrow \mathbb{E}^*_k B
\end{equation*}
\end{lemma}
\begin{proof}
Let~$A$ and~$B$ have universes~$\{a,b,c\}$ and~$\{a,b\}$ such that:
\begin{equation*}
    R^A(b,c) \mbox{ and } R^B(b,a)
\end{equation*}
is the only relation. Then the mapping:
\begin{align*}
    h(a) &= a\\
    h(b) &= b\\
    h(c) &= a
\end{align*}
is a homomorphism. We have prefix relation:
\begin{equation*}
    [a,b] \leq [a,b,c]
\end{equation*}
and so:
\begin{equation*}
    R^{\mathbb{E}^*_k A}([a,b], [a,b,c])
\end{equation*}
We consider the action of~$\mathbb{E}^*_k h$ on these sequences. Firstly the element wise operation preserves prefix ordering:
\begin{equation*}
    L_k [a,b] = [a,b] \leq [a,b,a] = L_k h [a,b,c]
\end{equation*}
The second normalization step does not:
\begin{equation*}
    \mathbb{E}^*_k [a,b] = [a,b] \not\leq [b,a] = \mathbb{E}^*_k [a,b,c]
\end{equation*}
Therefore~$h$ is a homomorphism, but~$\mathbb{E}^*_k h$ is not as it does not preserve~$R$.
\end{proof}

\section{Whack-A-Mole Normalization Scheme}
\label{sec:whack-a-mole}
This scheme is a rather ad-hoc attempt to patch up the deficiency in the plan of section~\ref{sec:naive}. The intuition is to:
\begin{enumerate}
    \item Remove duplicate elements from lists as in section~\ref{sec:naive}.
    \item Ensure that the final element is preserved by truncating back to the ``original final element''.
\end{enumerate}
Although the original problem seems to go away, this plan sounds like it will lead to whack-a-mole quite quickly, so maybe optimism possibly shouldn't be too high.

To formalize things, we define another auxiliary function:
\begin{definition}[Truncation]
\label{def:truncation}
For lists over a set~$X$, and~$x \in X$ we define the following \emph{truncation} function:
\begin{align*}
    t^x [] &= []\\
    t^x (y:ys) &=
    \begin{cases}
    y : t^x ys &\mbox{if } x \neq y\\
    [y] &\mbox{otherwise}
    \end{cases}
\end{align*}
We could define truncation at the first matching value from the right similarly. We shall denote this $r^x$ when needed in explanations.
\end{definition}
We the define our putative functor as follows:
\begin{definition}
For $\sigma$-structure~$A$, define~$\mathbb{E}^*_K$ as having:
\begin{description}
\item[Universe] The universe is the set of non-repeating, non-empty lists of elements from~$A$.
\item[Relations] For n-tuple $\overline{a}$ of elements, and n-ary relation symbol~$R$, we define:
\begin{equation*}
    R^{\mathbb{E}^*_k}(\overline{a})
\end{equation*}
if the elements of~$\overline{a}$ are pairwise comparable in the prefix order, and:
\begin{equation*}
    R^A(\epsilon(\overline{a}))
\end{equation*}
\end{description}
For homomorphism~$h : A \rightarrow B$ we define function:
\begin{equation*}
    \mathbb{E}^*_k(h)(xs:x) = t^{h x}(s^\emptyset \circ L^+_k h (xs:x))
\end{equation*}
Where we exploit shrinking and truncation from definitions~\ref{def:shrinking} and~\ref{def:truncation} respectively.
\end{definition}
\begin{definition}[Normalization]
It shall be convenient to define shorthand for the composite \emph{normalization} function:
\begin{equation*}
    \hat{n}(xs) = t^{\epsilon(xs)} \circ s^\emptyset
\end{equation*}
\end{definition}
We need a few technical lemmas to proceed.
\begin{lemma}
\label{lem:t-idem}
Truncation satisfies the following generalized idempotence property:
\begin{equation*}
t^{h a} \circ L^+_k h \circ t^a = t^{h a} \circ L^+_k h
\end{equation*}
\end{lemma}
\begin{lemma}
\label{lem:s-idem}
Shrinking satisfies the following generalized idempotence property:
\begin{equation*}
    s^\emptyset \circ L^+_k h \circ s^\emptyset = s^\emptyset \circ L^+_k h
\end{equation*}
\end{lemma}
\begin{lemma}
\label{lem:t-s-do-not-commute}
Shrinking and truncation commute:
\begin{equation*}
    s^\emptyset \circ t^a = t^a \circ s^\emptyset
\end{equation*}
Truncation from the right, $r^a$, and shrinking do not commute (Tom\'a\v{s}):
\begin{equation*}
    s^\emptyset \circ r^a = r^a \circ s^\emptyset
\end{equation*}
\begin{proof}
For the first step, this is essentially because both operations work from the same end
and preserve the order of first occurrences.
For the counterexample, consider input~$[a,b,a]$, and truncation at~$a$.
\end{proof}
We really only need a weaker property, which also conveniently involves less swapping during proofs
than using commutativity directly.
\begin{lemma}
\label{lem:t-s-interaction}
The following interaction between shrinking and truncation holds:
\begin{equation*}
    t^a \circ s^\emptyset \circ t^a = t^a \circ s^\emptyset
\end{equation*}
\end{lemma}
\begin{proof}
Both sides result in a non-repeating list of values in the input, in the order they first appear, and stopping at~$a$ if it appears.
\end{proof}
\end{lemma}
Normalization inherits generalized idempotence from its components.
\begin{lemma}
\label{lem:n-idem}
Normalization satisfies the following generalized idempotence property:
\begin{equation*}
    \hat{n} \circ L^+_k(h) \circ \hat{n} = \hat{n} \circ L^+_k(h)
\end{equation*}
\end{lemma}
\begin{proof}
    We calculate as follows:
    \begin{eqproof*}
    \hat{n} \circ L^+_k(h) \circ \hat{n}(xs)
    \explain{Definitions}
    t^{h\circ \epsilon(xs)} \circ s^\emptyset \circ L^+_k(h) \circ t^{\epsilon(xs)} \circ s^\emptyset(xs)
    \explain{Lemma \ref{lem:t-s-interaction}}
    t^{h\circ \epsilon(xs)} \circ s^\emptyset \circ t^{h\circ \epsilon(xs)} \circ L^+_k(h) \circ t^{\epsilon(xs)} \circ s^\emptyset(xs)
    \explain{Lemma \ref{lem:t-idem}}
    t^{h\circ \epsilon(xs)} \circ s^\emptyset \circ t^{h\circ \epsilon(xs)} \circ L^+_k(h) \circ s^\emptyset(xs)
    \explain{Lemma \ref{lem:t-s-interaction}}
    t^{h\circ \epsilon(xs)} \circ s^\emptyset \circ L^+_k(h) \circ s^\emptyset(xs)
    \explain{Lemma \ref{lem:s-idem}}
    t^{h\circ \epsilon(xs)} \circ s^\emptyset \circ L^+_k(h)
    \explain{Definitions}
    \hat{n} \circ L^+_k(h)
    \end{eqproof*}
\end{proof}
\begin{lemma}
\label{lem:s-monotone}
Shrinking is monotone with respect to prefix ordering:
\begin{equation*}
    x \leq y \Rightarrow s^\emptyset x \leq s^\emptyset y
\end{equation*}
\end{lemma}
Truncation is not monotone in a useful way, fortunately, we require less.
\begin{lemma}
\label{lem:t-pres-comparability}
Truncation preserves comparability in the prefix order, for any~$a$ and~$b$:
\begin{equation*}
    x \uparrow y \Rightarrow t^{a} x \uparrow t^{b} y
\end{equation*}
\end{lemma}
\begin{proof}
Any two prefixes of a prefix comparable pair are prefix comparable as they are linearly ordered.
\end{proof}
With these in place, we can now look at the action on morphisms of~$\mathbb{E}^*_k$.
\begin{lemma}
For $\sigma$-structure homomorphism~$h : A \rightarrow B$, $\mathbb{E}^*_k h$ is a homomorphism of type:
\begin{equation*}
    \mathbb{E}^*_k A \rightarrow \mathbb{E}^*_k B
\end{equation*}
\end{lemma}
\begin{proof}
Let~$\overline{a}$ be a tuple of elements in~$\mathbb{E}^*_k A$
Assume $R(\overline{a})$, and so, by definition $R(\epsilon\overline{a})$. As~$h$ is a homomorphism, we then have $R(h\epsilon\overline{a})$, which implies~$R(\epsilon\mathbb{E}^*_k(h)\overline{a})$.

The elements of~$\overline{a}$ are pairwise comparable. $L^+_k h$, $s^\emptyset$ and~$t$ preserve pairwise comparability by lemmas~\ref{lem:s-monotone} and~\ref{lem:t-pres-comparability}, and so the elements of~$\mathbb{E}^*_k(h)(\overline{a})$ are pairwise comparable. Therefore~$R(\epsilon\mathbb{E}^*_k(h)\overline{a})$ implies~$R(\mathbb{E}^*_k(h)\overline{a})$ by definition.
\end{proof}
\begin{lemma}
$\mathbb{E}^*_k$ is an endofunctor on the category of~$\sigma$-structures.
\end{lemma}
\begin{proof}
Preservation of identities is obvious. For composition we calculate:
\begin{eqproof*}
\mathbb{E}^*_k(k) \circ \mathbb{E}^*_k(h)(xs)
\explain{Definitions}
\mathbb{E}^*_k(k) \circ t^{h\epsilon(xs)} \circ s^\emptyset \circ L^+_k h (xs) 
\explain{Definitions}
\hat{n} \circ L^+_k k \circ \hat{n} \circ L^+_k h (xs)
\explain{Lemma \ref{lem:n-idem}}
\hat{n} \circ L^+_k k \circ  L^+_k h (xs)
\explain{Functoriality}
\hat{n} \circ L^+_k (k \circ h) (xs)
\explain{Definitions}
\mathbb{E}^*_k(k \circ h)(xs)
\end{eqproof*}
\end{proof}
We now making a couple of easy observations.
\begin{lemma}
Both $\epsilon_A$ and~$\delta_A$ are well defined homomorphisms.
\end{lemma}
\begin{proof}
Both preserve non-repeating non-empty lists. The remainder follows by noting they are restrictions of the version for the usual E-F comonads~$\mathbb{E}_k$.
\end{proof}
We require another technical lemma.
\begin{lemma}
\label{lem:tail-preservation}
Normalization preserves tail elements.
\begin{equation*}
   \epsilon \circ \hat{n} = \epsilon
\end{equation*}
\end{lemma}
\begin{proof}
Immediate from the definitions.
\end{proof}
\begin{lemma}
The tail maps $\epsilon$ are natural~$\mathbb{E}^*_k \Rightarrow 1$.
\end{lemma}
\begin{proof}
We calculate:
\begin{eqproof*}
\epsilon \circ \mathbb{E}^*_k(h)(xs)
\explain{Definitions}
\epsilon \circ \hat{n} \circ L^+_k(h)(xs)
\explain{Lemma~\ref{lem:tail-preservation}}
\epsilon \circ L^+_k(h)(xs)
\explain{Naturality w.r.t. $L^+_h$}
h \circ \epsilon
\end{eqproof*}
\end{proof}
Yet another technical lemma.
\begin{lemma}
\label{lem:normalization-commutation}
Normalization commutes with comultiplication as follows:
\begin{equation*}
    \delta \circ \hat{n} = \hat{n} \circ L^+_k(\hat{n}) \circ \delta
\end{equation*}
\end{lemma}
\begin{proof}
Essentially because~$\hat{n}$ takes the first appearances of everything in the list, and~$\delta$ preserves non-repeating sequences.
\end{proof}
\begin{lemma}
The prefixes map~$\delta$ is natural~$\mathbb{E}^*_k \Rightarrow \mathbb{E}^*_k \circ \mathbb{E}^*_k$.
\end{lemma}
\begin{proof}
We calculate as follows:
\begin{eqproof*}
\delta \circ \mathbb{E}^*_k(h)
\explain{Definition}
\delta \circ \hat{n} \circ L^+_k(h)
\explain{Lemma \ref{lem:normalization-commutation}}
\hat{n} \circ L^+_k(\hat{n}) \circ L^+_k(h)
\explain{Naturality w.r.t. $L^+_k$}
\hat{n} \circ L^+_k(\hat{n}) \circ (L^+_k \circ L^+_k)(h) \circ \delta
\explain{Functoriality}
\hat{n} L^+_k (\hat{n} \circ L^+_k{h}) \circ \delta
\explain{Definitions}
(\mathbb{E}^*_k \circ \mathbb{E}^*_k)(h) \circ \delta
\end{eqproof*}
\end{proof}
It remains to check the 3 axioms for a comonad on comonoid form.
\begin{lemma}
The left counit axiom holds:
\begin{equation*}
    \epsilon_{\mathbb{E}^*_k} \circ \delta = 1
\end{equation*}
\end{lemma}
\begin{proof}
Identical to that for~$\mathbb{E}_k$.
\end{proof}
\begin{lemma}
The right counit axiom holds:
\begin{equation*}
    \mathbb{E}^*_k(\epsilon) \circ \delta = 1
\end{equation*}
\end{lemma}
\begin{proof}
For a non-repeating input $xs$, $\delta(xs)$ is non-repeating. $\mathbb{E}^*_k(\epsilon)$ then recovers the original list, which is still non-repeating, and so normalization does nothing, and~$xs$ is returned.
\end{proof}
\begin{lemma}
The comultiplication axioms holds:
\begin{equation*}
    \delta_{\mathbb{E}^*_k} \circ \delta = \mathbb{E}^*_k(\delta) \circ \delta
\end{equation*}
\end{lemma}
\begin{proof}
We note that this axiom holds for~$\mathbb{E}_k$ as it is a comonad. The equation above only involves non-repeating sequences at all stages, so normalization does nothing. Therefore the claim reduces to the ordinary E-F comonad property.
\end{proof}
So apparently we're good as this is a legitimate comonad. Hmmm. We continue anyway.
\begin{lemma}
For a $\sigma$-structure $A$, normalization is a~$\sigma$-structure homomorphism of type~$\mathbb{E}_k(A) \rightarrow \mathbb{E}^*_k(A)$.
\end{lemma}
\begin{proof}
Assume~$R^{\mathbb{E}_k(A)}(\overline{a})$, so the elements of~$\overline{a}$ are pairwise comparable, and~$R^A(\epsilon(\overline{a}))$. Then~$\hat{n}$ preserves pairwise comparability by lemmas~\ref{lem:s-monotone} and~\ref{lem:t-pres-comparability}. It also preserves final elements. Therefore~$R^{\mathbb{E}^*_k}(\hat{n}\overline{a})$.
\end{proof}
\begin{lemma}
\label{lem:n-natural}
Normalization is natural~$\mathbb{E}_k \Rightarrow \mathbb{E}^*_k$.
\end{lemma}
\begin{proof}
We calculate:
\begin{eqproof*}
\mathbb{E}^*_k(h) \circ \hat{n}
\explain{Definitions}
\hat{n} \circ L^+_k(h) \circ \hat{n}
\explain{Lemma~\ref{lem:n-idem}}
\hat{n} \circ L^+_k(h)
\explain{definitions}
\hat{n} \circ \mathbb{E}_k(h)
\end{eqproof*}
\end{proof}
\begin{lemma}
Normalization is a comonad morphism~$\mathbb{E}_k \Rightarrow \mathbb{E}^*_k$.
\end{lemma}
\begin{proof}
Lemma~\ref{lem:n-natural} establishes naturality. Lemmas~\ref{lem:tail-preservation} and~\ref{lem:normalization-commutation} complete the proof.
\end{proof}

\end{document}
